% $Header: /cvsroot/latex-beamer/latex-beamer/solutions/conference-talks/conference-ornate-20min.en.tex,v 1.7 2007/01/28 20:48:23 tantau Exp $
\documentclass[14pt]{beamer}

\newcommand{\strong}[1]{{\normalfont\fontseries{b}\selectfont #1}}
\newcommand{\class}[1]{\mbox{\textsf{#1}}}
\newcommand{\func}[1]{\mbox{\texttt{#1()}}}
\newcommand{\code}[1]{\mbox{\texttt{#1}}}
\newcommand{\pkg}[1]{\strong{#1}}
\newcommand{\samp}[1]{`\mbox{\texttt{#1}}'}
\newcommand{\proglang}[1]{\textsf{#1}}
\newcommand{\putat}[3]{\begin{picture}(0,0)(0,0)\put(#1,#2){#3}\end{picture}}


\usepackage{beamerthemeAmsterdam}
\usepackage{relsize}
\usepackage{natbib}
\renewcommand{\bibsection}{\subsubsection*{\bibname } }
\def\newblock{}

\mode<presentation>
{
  \usetheme{Amsterdam}
}


\usepackage[english]{babel}

\usepackage[latin1]{inputenc}

\usepackage{times}
\usepackage[T1]{fontenc}
\usepackage[absolute,overlay]{textpos}
\newenvironment{reference}[2]{
  \begin{textblock*}{\textwidth}(#1,#2)
    \footnotesize\it\bgroup\color{black!50!black}}{\egroup\end{textblock*}}

\usepackage{graphics}
% Or whatever. Note that the encoding and the font should match. If T1
% does not look nice, try deleting the line with the fontenc.


%\title{Babelstream\smaller{\smaller{\smaller{{\texttrademark}}}}: 
\title{Statistical Consulting (BIS 578) Lecture 1}


\author{Michael J. Kane}

\date{}

\begin{document}

\begin{frame}
  \titlepage
\end{frame}

\begin{frame}{Outline for today's class}
  \tableofcontents
%  \let\thefootnote\relax\footnotetext{\relsize{-3}{The Research Plan and 
%    Plan of Study sections are adapted from slides provided by James Dziura  \\
%    }}
\end{frame}

\section{Changing the Class Time}

\subsection*{}

\begin{frame}{Do you need to get to Regression Class?  Then go!}
\begin{itemize}
\item This class is going to be organizational only
\item The slides will be posted on classesv2
\item There is a short homework that is due at the end of next
  week.  See the slides for more details
\item We are going to propose a new time for the class and
  I'll send out an email for everyone to vote on it.
\end{itemize}
\end{frame}

\begin{frame}{}
\begin{center}
\larger{OK, now that they are gone, let's propose class times.}
\end{center}
\end{frame}

\section{About the Statistical Consulting Class}

\subsection*{}

\begin{frame}{Which class is this?}
\begin{center}
\larger{This is BIS 578, Statistical Consulting}
\end{center}
\end{frame}

\begin{frame}{Why should you take this class?}
This class offers the chance for students to gain experience and practical 
knowledge working as a statistical consultant in a ``real-world'' setting. 
\end{frame}

\begin{frame}{What are the objectives of the class?}
\begin{enumerate}
\item Work with an investigator to understand the goals of a clinical study.
\item Contribute to a clinical study by designing and implementing statistical 
  tests that answer questions posed by the study.
\item Gain experience in data cleaning, exploring, and analyzing a 
  ``real-world'' data sets
\item Present the results of your statistical investigations in reports 
  and presentations.
\end{enumerate}
\end{frame}

\begin{frame}{How will the class be taught?}
\begin{itemize}
\smaller{\smaller{
\item The first 3/4 classes will be taught by Peter Peduzzi, Jim Dzuira, and me.
\item At the $4^\text{th}$ week you'll choose a project to work on.
\item Each project has an associated facilitator (Peter Peduzzi, Jim Dzuira,
  Veronika Northrup, or me) and an associated investigator.
\item For the following weeks you'll present your (or your team's) progress
  for your facilitator, investigator, and other students.
\item At the end of your project(s) you'll submit a write-up and present
  your work.
\item During the semester you'll also attend consultations between members
  of the Yale Center for Analytical Sciences and investigators.
}}
\end{itemize}
\end{frame}

\begin{frame}{How will the class be graded?}
\begin{tabular}{|l|c|}\hline 
Participation & 10\% \\ \hline
Clinic write-ups & 40\% \\ \hline
Project write-up and presentation & 50\% \\ \hline
\end{tabular}
\end{frame}

\begin{frame}{Where are the materials for the class kept?}
Slides, homework, the syllabus, and other materials directly related
to the class will be kept at:

\begin{center}
\url{https://classesv2.yale.edu/portal}
\end{center}

\LaTeX \ and Beamer code to generate documents can be found at:
\begin{center}
\url{https://github.com/kaneplusplus}
\end{center}
\end{frame}

\begin{frame}
\begin{center}
\larger{\larger{Are there other questions?}}
\end{center}
\end{frame}

\section{Homework}

\subsection*{}

\begin{frame}{Your first homework assignment}
Get HIPAA certified
\begin{enumerate}
\item Go to \url{http://learn.caim.yale.edu/hipaaTraining/}
\item Select the ``HIPAA for Research Staff'' course
\item Complete the course with a passing grade
\item Upload the certificate of completion to classesv2:
  \url{BIS 578 01 (F11) Drop Box/HIPAA certifications}
\item This will be due 9/10
\end{enumerate}
\end{frame}

\end{document}


