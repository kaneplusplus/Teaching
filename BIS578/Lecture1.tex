% $Header: /cvsroot/latex-beamer/latex-beamer/solutions/conference-talks/conference-ornate-20min.en.tex,v 1.7 2007/01/28 20:48:23 tantau Exp $
\documentclass[13pt]{beamer}

\newcommand{\strong}[1]{{\normalfont\fontseries{b}\selectfont #1}}
\newcommand{\class}[1]{\mbox{\textsf{#1}}}
\newcommand{\func}[1]{\mbox{\texttt{#1()}}}
\newcommand{\code}[1]{\mbox{\texttt{#1}}}
\newcommand{\pkg}[1]{\strong{#1}}
\newcommand{\samp}[1]{`\mbox{\texttt{#1}}'}
\newcommand{\proglang}[1]{\textsf{#1}}
\newcommand{\putat}[3]{\begin{picture}(0,0)(0,0)\put(#1,#2){#3}\end{picture}}


\usepackage{beamerthemeAmsterdam}
\usepackage{setspace}
\usepackage{listings}
\usepackage{multirow}
\usepackage{movie15}
\usepackage{relsize}
\lstset{basicstyle=\ttfamily}
%\lstset{language=R}
\lstset{tabsize=2}
\lstset{showstringspaces=false}

%\{onehalfspacing}
%\onehalfspacing
\usepackage{Sweave}

% This file is a solution template for:

% - Talk at a conference/colloquium.
% - Talk length is about 20min.
% - Style is ornate.



% Copyright 2004 by Till Tantau <tantau@users.sourceforge.net>.
%
% In principle, this file can be redistributed and/or modified under
% the terms of the GNU Public License, version 2.
%
% However, this file is supposed to be a template to be modified
% for your own needs. For this reason, if you use this file as a
% template and not specifically distribute it as part of a another
% package/program, I grant the extra permission to freely copy and
% modify this file as you see fit and even to delete this copyright
% notice. 


\mode<presentation>
{
%  \usetheme{Montpellier}
  \usetheme{Amsterdam}

%  \usetheme{Antibes}
  % or ...

  % \setbeamercovered{transparent}
  % or whatever (possibly just delete it)
}


\usepackage[english]{babel}
% or whatever

\usepackage[latin1]{inputenc}
% or whatever

\usepackage{times}
\usepackage[T1]{fontenc}
\usepackage[absolute,overlay]{textpos}
\newenvironment{reference}[2]{
  \begin{textblock*}{\textwidth}(#1,#2)
    \footnotesize\it\bgroup\color{black!50!black}}{\egroup\end{textblock*}}

\usepackage{graphics}
% Or whatever. Note that the encoding and the font should match. If T1
% does not look nice, try deleting the line with the fontenc.


%\title{Babelstream\smaller{\smaller{\smaller{{\texttrademark}}}}: 
\title{BIS 578 Lecture 1}



\author{Michael J. Kane}
% - Give the names in the same order as the appear in the paper.
% - Use the \inst{?} command only if the authors have different
%   affiliation.

\date{}

%\institute
%{
%  \inst{1}
%  Yale Center for Analytical Sciences, Yale University
%  \and 
%  \inst{2}
%  Department of Statistics, Yale University
%}

% - Use the \inst command only if there are several affiliations.
% - Keep it simple, no one is interested in your street address.

% - Either use conference name or its abbreviation.
% - Not really informative to the audience, more for people (including
%   yourself) who are reading the slides online

%\subject{Theoretical Computer Science}
% This is only inserted into the PDF information catalog. Can be left
% out. 



% If you have a file called "university-logo-filename.xxx", where xxx
% is a graphic format that can be processed by latex or pdflatex,
% resp., then you can add a logo as follows:

% \pgfdeclareimage[height=0.5cm]{university-logo}{university-logo-filename}
% \logo{\pgfuseimage{university-logo}}


% Delete this, if you do not want the table of contents to pop up at
% the beginning of each subsection:
%\AtBeginSubsection[]
%{
%  \begin{frame}<beamer>{Outline}
%    \tableofcontents[currentsection,currentsubsection]
%  \end{frame}
%}
%\AtBeginSection[]
%{
%  \begin{frame}<beamer>{Outline}
%    \tableofcontents[currentsection,currentsubsection]
%  \end{frame}
%}



% If you wish to uncover everything in a step-wise fashion, uncomment
% the following command: 

%\beamerdefaultoverlayspecification{<+->}


\begin{document}

\begin{frame}
  \titlepage
%  \begin{reference}{10mm}{78mm}
%    This is an expanded version of a talk given I gave at the 2010 
%    \proglang{R}Finance conference. 
%    The original can be downloaded from Bryan Lewis's website at 
%    \url{http://illposed.net/LewisKaneRInFinance.pdf}.
%  \end{reference}
\end{frame}

\begin{frame}{Outline}
  \tableofcontents
\end{frame}

% Structuring a talk is a difficult task and the following structure
% may not be suitable. Here are some rules that apply for this
% solution: 

% - Exactly two or three sections (other than the summary).
% - At *most* three subsections per section.
% - Talk about 30s to 2min per frame. So there should be between about
%   15 and 30 frames, all told.

% - A conference audience is likely to know very little of what you
%   are going to talk about. So *simplify*!
% - In a 20min talk, getting the main ideas across is hard
%   enough. Leave out details, even if it means being less precise than
%   you think necessary.
% - If you omit details that are vital to the proof/implementation,
%   just say so once. Everybody will be happy with that.

\section{Developing a research question with a clinician}

\begin{frame}{Know you role}
\end{frame}

\section{The sky isn't falling... is it?}

\subsection*{}

\begin{frame}{The R Project}
\larger{
\begin{itemize}
\item 18 years and counting... with roots extending 35+ years
\item Over 4200 packages on Bioconductor and CRAN
\item Tiobe Programming Community Index rank of 34, just behind Matlab (26)
 and SAS (28): \url{http://www.tiobe.com/}
\item Aguably the most popular language for research in statistical computing
\item R does what we need 99\% of the time
\end{itemize}
}
\end{frame}

\begin{frame}{}
\begin{center}
\larger{\larger{So, why are we having this session?}}
\end{center}
\end{frame}

\begin{frame}{Topics at the frontier of statistical computing}
\larger{\larger{
\begin{itemize}
\item Bytecode support
\item Threading 
\item Reflection
\item 64-bit indexing
\item Performance
\end{itemize}
}}
\end{frame}

\begin{frame}{Bytecode and Virtual Machine Support}
\larger{\larger{
Unless he was delayed by a committee meeting earlier this morning,
we just heard from Luke about the R bytecode compiler.

\vspace*{1cm}

Progress has been made!
}}
\end{frame}

\begin{frame}{Threading}
\larger{\larger{
We can't spawn threads in R

\vspace*{1cm}

We can make use of threaded code 
\begin{itemize}
\item via Luke's pnmath extension \cite{pnmath}
\item via multi-threaded BLAS and LAPACK \cite{RInstallAndAdmin}
\end{itemize}
}}
\end{frame}

\begin{frame}{Do we need threads?}
\larger{
\begin{itemize}
\item
Process parallelism often trumps thread parallelism \cite{EmailJustin},
because processes control:

\begin{itemize}
\larger{
\item memory allocation locks
\item mmap page-fault locks 
}
\end{itemize}

\item Parallel computing is well-supported in R: 
  16 different packages on CRAN address
  parallel computing via process parallelism
\end{itemize}

\vspace*{0.5cm}
Maybe we don't need it, but thread support would be nice.

}
\end{frame}

\begin{frame}[fragile]{Reflection}
\larger{
Objects that are external pointers (or non-native R objects) can't be copied directly:
\begin{Schunk}
\begin{Sinput}
> library(bigmemory)
> x <- big.matrix(2, 2, init=0)
> y <- x
> x[1,1] <- 99
> y[,]
\end{Sinput}
\begin{Soutput}
     [,1] [,2]
[1,]   99    0
[2,]    0    0
\end{Soutput}
\end{Schunk}
}
\end{frame}

\begin{frame}{Updating R for 64-bit indexing}
\larger{\larger{
\begin{itemize}
\item About 430,000 lines of code in R src directory ($\sim$370,000 lines in .c 
  files, $\sim$60,000 lines in .h files)
\vspace*{0.5cm}
\item Very difficult and time-consuming \cite{RInternals}
\vspace*{0.5cm}
\item Limited or no academic currency
\end{itemize}
}}
\end{frame}

\begin{frame}{Performance}
\larger{
However, low-level benchmarks performed by Justin Talbot in late 
2010 \cite{EmailJustinBench} show that compared to C:
%\vspace*{0.5cm}
\begin{itemize}
\item Scalar operations: 
\begin{itemize}
\larger{
\item About 40\% of overhead comes from traversing the abstract syntax tree (AST)
\item About 60\% comes from memory management
}
\end{itemize}
%\vspace*{0.5cm}
\item Sequences of vectorized operations in R are about 10 times slower than
hardware capabilities (e.g. \texttt{2*(x+y)})
\end{itemize}}

\vspace*{0.5cm}

R's performance is good but there may be room for improvement.
\end{frame}

\begin{frame}{}

\begin{center}
\larger{\larger{The sky isn't falling.  Yet.}}
\end{center}
\end{frame}

\section{Overview: New directions for R}

\subsection*{}

\begin{frame}{Current efforts on the frontier represented today}
\larger{\larger{
\begin{itemize}
\item Continue development of R (Luke and R Core)
\vspace*{0.5cm}
\item Migrate R to C++, preserve the syntax (Andrew)
\vspace*{0.5cm}
\item Start from scratch, preserve the syntax (Simon, Mike and Jay)
\end{itemize}
}}
\end{frame}

\begin{frame}{Some efforts not directly represented here today}
\larger{
\begin{itemize}
\item Omegahat (Duncan Temple Lang \cite{omegahat})
\item Start from scratch, change the grammar (Duncan Temple-Lang and Ross 
  Ihaka \cite{backtofuture})
\item R on the JVM (Alexander Bertram, Peter Robinette, Michael Williams \cite{renjin})
\item Riposte, a bytecode interpreter for R based on Intel's ARBB package (Justin Talbot \cite{riposte}) 
\item A JIT for R (Jan Vitek \cite{vitek})
\item A new R-like language (Ross Ihaka and Brendan McCardle \cite{ross})
\item A Lisp-based system (Tony Rossini \cite{rossini})
\end{itemize}
}
\end{frame}

\section{Explorations with the Parrot Virtual Machine}

\subsection*{}

\begin{frame}
\begin{center}
\larger{\larger{The Not Quite R (NQR) Project: \\ \vspace*{0.5cm}
Explorations Using \\ \vspace*{0.5cm}
the Parrot Virtual Machine}}
\end{center}
%  \titlepage
%  \begin{reference}{10mm}{78mm}
%    This is an expanded version of a talk given I gave at the 2010 
%    \proglang{R}Finance conference. 
%    The original can be downloaded from Bryan Lewis's website at 
%    \url{http://illposed.net/LewisKaneRInFinance.pdf}.
%  \end{reference}
\end{frame}

\begin{frame}{What is the Parrot Virtual Machine?}
\larger{\begin{itemize}
\item ``Parrot is a virtual machine designed to efficiently compile 
  and execute bytecode for dynamic languages.'' \url{-- http://www.parrot.org}
\vspace*{0.3cm}
\item Formally started by the Perl community around 2001
\vspace*{0.3cm}
\item Parrot Foundation created in 2008
\vspace*{0.3cm}
\item Includes a suite of tools for quickly developing new high-level
  languages and compilers 
\vspace*{0.3cm}
\item Provides high-level language interoperability
\end{itemize}}
\end{frame}



\begin{frame}{Which languages are currently supported on Parrot?}
\larger{Actively developed and stable:
\begin{itemize}
\item Rakudo Perl 6
\item Parrot Lua
\item Winxed
\item nqp
\item C/C++ through a native call interface (NCI)
\end{itemize}}

\vspace{0.5cm}

About 25 other languages in various stages of development
\end{frame}

\begin{frame}{Why develop a language for Parrot?}
\larger{
\begin{itemize}
\item Language interoperability
\vspace*{0.3cm}
\item Provides a full-featured assembly language:
\begin{itemize}
\larger{
\item No need to re-implement arrays, hashes, ...
\item Parrot collects the garbage for us
}
\end{itemize}
\vspace*{0.3cm}
\item Implementing a language is a matter of mapping the grammar
  to the existing constructs of the compiler toolkit
\vspace*{0.3cm}
\item JIT on the horizon
\vspace*{0.3cm}
\item An active, friendly, and helpful developer community
\end{itemize}}
\end{frame}


\begin{frame}{Exploration with the Parrot Virtual Machine}
\larger{
\begin{itemize}
\item We designed and Jay implemented a system supporting
a subset of the S syntax
using the Parrot Virtual Machine.
\vspace*{0.3cm}
\item Code available: \url{https://github.com/NQRCore/}
\vspace*{0.3cm}
\item NQR stands for ``Not Quite R'' where ``Not Quite'' is an understatement.
\vspace*{0.3cm}
\item Some things don't work: It's Jay's fault.
\end{itemize}}
\end{frame}



\begin{frame}{NQR on the Parrot Virtual Machine}
\larger{\larger{
\begin{itemize}
\item Support the core S syntax including vectors of
Integer, Float, and String.
\vspace*{0.5cm}
\item Leverage existing libraries like the
GNU Scientific Library or R's libRmath.so
\vspace*{0.5cm}
\item Make initial design decisions consistent with a longer-term
``scalable'' model
\end{itemize}
}}
\end{frame}

\begin{frame}[fragile]{Benchmark: a trivial while loop}
\larger{
Note from Jay: I haven't yet mastered the \texttt{for} loop, sorry.
\vspace*{0.5cm}
\begin{Schunk}
\begin{Sinput}
N <- 1000
while (N > 0) {
  N = N - 1
}
\end{Sinput}
\end{Schunk}
}
\end{frame}

\begin{frame}{Benchmark: a trivial while loop}
\begin{figure}
\includegraphics[width=4in, height=3in]{while.pdf}
\end{figure}
\end{frame}

\begin{frame}[fragile]{Benchmark: mean of random exponentials}
\larger{
Notes from Jay: 
\begin{itemize}
\item \texttt{rexp()} in NQR is not vectorized as in R and pays the
price of a non-optimized loop 
\item \texttt{mean()} uses
the GNU Scientific Library implementation.
\end{itemize}

\vspace*{0.5cm}
\begin{Schunk}
\begin{Sinput}
N <- 1000
set.seed(1,2)
foo <- mean(rexp(N, 1.0))
\end{Sinput}
\end{Schunk}
}
\end{frame}

\begin{frame}{Benchmark: mean of random exponentials}
\begin{figure}
\includegraphics[width=4in, height=3in]{meanexp.pdf}
\end{figure}
\end{frame}

\begin{frame}{NQR's Potential on Parrot VM}
\larger{
\begin{itemize}
\item Bytecode support: yes
\vspace*{0.3cm}
\item Threading: in Parrot pipeline
\vspace*{0.3cm}
\item Reflection: yes, by design, with language interoperability
\vspace*{0.3cm}
\item 64-bit indexing: yes
\vspace*{0.3cm}
\item Performance: Naive NQR won't beat the compiled C performance of R,
but will continue to improve as Parrot evolves.
%not impressive in interactive mode, but code can
%be compiled and JIT is on the way for Parrot.
\end{itemize}
}
\end{frame}

\begin{frame}{Future mucking about in the sandbox}
\larger{
\begin{itemize}
\item Refine/debug core language with vectors only
\vspace*{0.3cm}
\item Add memory-mapped files for larger-than-RAM objects for seamless
scalability
\vspace*{0.3cm}
\item Add lists (a basic hash already exists)
\vspace*{0.3cm}
\item Add classes for \texttt{matrix} and \texttt{data.frame}
\vspace*{0.3cm}
\item Add \texttt{read.csv()} so we can use real data
\vspace*{0.3cm}
\item Explore graphics
\end{itemize}
}
\end{frame}

\begin{frame}{Want to play with it?}
\larger{
\larger{
\begin{center}
\url{https://github.com/NQRCore}
\end{center}
}}

\vspace*{0.5cm}
You'll need:
\begin{itemize}
\item
Parrot \url{http://www.parrot.org}
\item
libffi (this dependency will be phased out)
\item GNU Scientific Library \url{http://www.gnu.org/software/gsl/}
\item Jay has only tested in Linux; MacOS should be fine.
\item Windows?  In theory, yes (Parrot attempts to support Windows and more).
\end{itemize}
\end{frame}

\begin{frame}[fragile]{Appendix: NQR syntax examples}
\smaller{
\begin{Schunk}
\begin{Sinput}
jay@bayesman:~/Desktop/NQR$ ./installable_nqr 

Not Quite R for Parrot VM, Version 0.0.9, July 31, 2011.

To exit, use <ctrl>-D.
Please see t/00-sanity.t for currently-supported syntax.

> a <- 1000 + 100 * rexp(10, 1.2345)
> print(c(mean(a), sd(a), min(a), max(a)))
\end{Sinput}
\begin{Soutput}
1110.205677 122.5590509 1003.827809 1334.90338
\end{Soutput}
\begin{Sinput}
> b <- sort(a)
> print(paste("Min two different ways:",
              a[which.min(a)], b[0]))
\end{Sinput}
\begin{Soutput}
Min two different ways: 1003.827809 1003.827809
\end{Soutput}
\end{Schunk}
}
\end{frame}


\setbeamertemplate{bibliography item}[text]

\begin{frame}[allowframebreaks]
  \frametitle{References}
  \begin{thebibliography}{9}
    \bibitem{renjin}
    A. Bertram, P. Robinette, and M. Williams.
    \newblock{R on the JVM.
      \url{http://code.google.com/p/renjin}.}
    \bibitem{ross}
    R. Ihaka.
      \newblock{\url{http://www.stat.auckland.ac.nz/~ihaka/}.}
    \bibitem{backtofuture}
    R. Ihaka and D. Temple Lang.
    \newblock{Back to the Future: Lisp as a Base for a Statistical Computing
      System.}
    \newblock {\em CompStat}, August 25, 2008. 
    \bibitem{RInstallAndAdmin}
    The R Installation and Administration Manual, Section A.3.1.4, 2011
      \newblock{\url{www.r-project.org/doc/manuals/R-admin.html}.}
    \bibitem{RInternals}
    The R Internals Manual, Section 11, 2011
      \newblock{\url{www.r-project.org/doc/manuals/R-admin.html}.}
    \bibitem{rossini}
    T. Rossini.
    \newblock{Github repository.
      \url{https://github.com/blindglobe}.}
    \bibitem{EmailJustinBench}
    J. Talbot 
    \newblock{(personal communication, December 19, 2010).}
    \bibitem{EmailJustin}
    J. Talbot 
    \newblock{(personal communication, December 22, 2010).}
    \bibitem{riposte}
    J. Talbot.
    \newblock{Riposte, a bytecode interpreter for R.
      \url{https://github.com/jtalbot/riposte}.}
    \bibitem{omegahat}
    D. Temple Lang. 
    \newblock{The Omegahat Project.
      \url{www.omegahat.org}.}
    \bibitem{pnmath}
    L. Tierney. 
    \newblock{The pnmath package for R.
      \url{www.stat.uiowa.edu/~luke/R/experimental/}.}
    \bibitem{iethreading}
    L. Tierney.
    \newblock{Implicit and explicit parallel computing in R}
    \newblock{COMPSTAT 2008: Proceedings in Computation Statistics}, 42--51.
    2008.
    \bibitem{vitek}
    J. Vitek.
    \newblock{JIT grant.
      \url{http://www.cs.purdue.edu/people/faculty/jv/}.}
  \end{thebibliography}
\end{frame}

\end{document}


