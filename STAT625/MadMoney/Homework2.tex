\documentclass[12pt]{article}

\usepackage{graphics}
\usepackage{graphicx}
\usepackage{Sweave}
\usepackage{accents}

% New from euler:
\usepackage{ae}
\usepackage{color}
\usepackage{url}
\usepackage{hyperref}
\usepackage{setspace}

\topmargin=-0.5in
\textheight=9in
\textwidth=6.5in
\oddsidemargin=0in

%\usepackage{CJK}
%\usepackage{pinyin}
\def\E{\mathord{I\kern-.35em E}}
\def\R{\mathord{I\kern-.35em R}}
\def\P{\mathord{I\kern-.35em P}}
\def\I{\mathord{1\kern-.35em 1}}
\def\wt{\mathord{\widehat{\theta}}}

\newcommand{\proglang}[1]{\textbf{#1}}
\newcommand{\pkg}[1]{\texttt{\textsl{#1}}}
\newcommand{\code}[1]{\texttt{#1}}
\newcommand{\mg}[1]{{\textcolor {magenta} {#1}}}
\newcommand{\gr}[1]{{\textcolor {green} {#1}}}
\newcommand{\bl}[1]{{\textcolor {blue} {#1}}}

\newtheorem{thm}{Theorem}[section]
\newtheorem{myexplore}[thm]{Explore}
\newtheorem{mybackground}[thm]{Background}
\newtheorem{myquestion}[thm]{Question}
\newtheorem{myexample}[thm]{Example}
\newtheorem{mydefinition}[thm]{Definition}
\newtheorem{mytheorem}[thm]{Theorem}

\pagestyle{myheadings}    % Go for customized headings
\markboth{notused left title}{Michael Kane, Department of Statistics, Yale University \copyright 2010}
\newcommand{\sekshun}[1]                % In 'article' only the page
        {                               % number appears in the header.
        \section{#1}                    % I want the section name AND
        \markboth{#1 \hfill}{#1 \hfill} % the page, so I need a new kind
        }                               % of '\sekshun' command. 


\begin{document}

\doublespace

\setkeys{Gin}{width=1.0\textwidth} 



\begin{center}

%%%%%%%%%%%%%%%%%%%%%%%%%%%%%%%%%%%%%%%%%%%%
%%%%%%%%%%%%%%%%%%%%%%%%%%%%%%%%%%%%%%%%%%%%

{\Large\bf STAT 625: Case Studies\\
\vspace*{0.5cm}

%%%%%%%%%%%%%%%%%%%%%%%%%%%%%
Homework Due Thursday, September 23\\

%%%%%%%%%%%%%%%%%%%%%%%%%%%%%

}

%%%%%%%%%%%%%%%%%%%%%%%%%%%%%%%%%%%%%%%%%%%%
%%%%%%%%%%%%%%%%%%%%%%%%%%%%%%%%%%%%%%%%%%%%


\end{center}

\begin{raggedright}
\parindent=0.5in

%%%%%%%%%%%%%%%%%%%%%%%%%%%%%%%%%%%%%%%%%%%%%%%%%%%%%%%%%%%%%%%%%%%%%%%%%%%%%%%%%%%%%%%%%%%%%%%%

\section{The \proglang{Python} Challenge... in \proglang{R}}

The \href{http://www.pythonchallenge.com}{Python Challenge} website contains
a series of ``programming riddles'' that provide an entertaining way
of exploring the \proglang{Python} programming language.  
However, the challenges are not \proglang{Python} specific and 
any modern programming language can be used to create a solution,
including \proglang{R}.
For Homework 2, solve challenges 1 through 4 using \proglang{R} and 
write a short report describing 
your solutions.  You may work in teams of 2 and you should be ready to 
describe your solutions on the due date, September 23.

%%%%%%%%%%%%%%%%%%%%%%%%%%%%%%%%%%%%%%%%%%%%%%%%%%%%%%%%%%%%%%%%%%%%%%%%%%%%%%%%%%%%%%%%%%%%%%%%

\end{raggedright}

\end{document}
